\documentclass[xcolor=dvipsnames]{beamer}
%\usecolortheme[named=Brown]{structure}
\useoutertheme{infolines}
\usetheme[height=7mm]{Rochester} 
\setbeamertemplate{navigation symbols}{} 
\setbeamertemplate{items}[ball]
\usepackage{xeCJK}
\usepackage{makeidx}
\usepackage{multirow}
\usepackage{color}
\usepackage{graphicx}
\setCJKmainfont{AR PL UKai CN}

% 将日期变为中文格式
\renewcommand{\today}{\number\year 年 \number\month 月 \number\day 日}

\begin{document}

\title{Python初探}
\author{彭鹏}
\institute{汉仪通信}

\date{\today}
\begin{frame}
    \titlepage
\end{frame}

\begin{frame}
    \frametitle{Outline}
    \tableofcontents[pausesections]
\end{frame}


\begin{frame}
    \frametitle{大纲}
    \begin{itemize}[<+->]
        \item
            Python介绍
        \item
            Python基本使用
        \item
            Python实例
        \item
            参考资料
    \end{itemize} 
\end{frame}

\begin{frame}
    \frametitle{Python介绍--历史}
    Python是自由软件的丰硕成果之一
    \begin{itemize}[<+->]
        \item 
            \alert{创始人}是Guido van Rossum
        \item
            \alert{时间地点}是1989圣诞期间与阿姆斯特丹
        \item
            \alert{名字来源}与创始人是大蟒蛇飞行马戏团爱好者
        \item
            \alert{发展渊源}从ABC发展而来主要受Modula-3的影响结合了Unix shell和C的习惯
    \end{itemize} 
\end{frame}

\begin{frame}
    \frametitle{Python介绍--特点与用途}
    Python是一种面向对象、直译式语言 
    
    Python PK C/C++
    \begin{itemize}[<+->]
        \item 
            \alert{优点}有简单易学、跨平台、标准库全面内建强大的数据结构(列表、字典)自动管理内存、易于扩展、解释执行 免费、开源
        \item
            \alert{缺点}有相对于C/C++运行速度慢
        \item
            \alert{用途}除性能要求严苛的应用(例如实时kernel)外的所有用途,如:原型开发、科学计算、日常管理、GUI、游戏开发等。
    \end{itemize} 
\end{frame}

\begin{frame}
    \frametitle{Python介绍--安装}
    Python是每个平台都使用平台原声的安装方式,这里以Linux及Windows平台举例
    \begin{itemize}[<+->]
        \item
            Ubuntu(Linux)
            
            sudo apt-get intall python3 
        \item
            Windowx

            64位机:http://www.python.org/ftp/python/3.3.2/python-3.3.2.amd64.msi
            32位机:http://www.python.org/ftp/python/3.3.2/python-3.3.2.msi
            下载完成后,Next到Finish即可
    \end{itemize} 
\end{frame}

\begin{frame}
    \frametitle{Python介绍--学习方法}
    Python开发\alert{八荣八耻}
    \pause 

    以动手实践为荣,以只看不练为耻。
    \pause 

    以打印日志为荣,以单步跟踪为耻。
    \pause 

    以空白分隔为荣,以制表分隔为耻。
    \pause 

    以单元测试为荣,以手工测试为耻。
    \pause 

    以代码重用为荣,以复制粘贴为耻。
    \pause 

    以多态应用为荣,以分支判断为耻。
    \pause 

    以Pythonic为荣,以冗余拖沓为耻。
    \pause 

    以总结思考为荣,以不求甚解为耻。
\end{frame}

\begin{frame}
    \frametitle{Python基本使用--启动}
    Python有命令行、脚本两种启动方式
    \begin{itemize}[<+->]
        \item 命令行:
            
            Linux:   在终端中键入 python
            Windows: 在dos提示符中键入 python
        \item 脚本:

            将Python代码存入脚本文件,然后在命令行模式中使用脚本文件名作为参数启动python如: python hello.py
    \end{itemize} 
    \only<2->{当然啦,GUI系统中双击也是可以滴……}
\end{frame}

\begin{frame}
    \frametitle{Python基本使用—变量、表达式、语句}
    \begin{itemize}[<+->]
        \item 变量赋值: 
            \begin{semiverbatim} 
            % 如何两边对其
            a = 1           a是一个整形,值为1

            b = "test"      b是一个字符串,值为"test"

            a,b = 1,"test"  \alert{多元赋值},与前两句等效
            \end{semiverbatim} 
        \item 表达式和运算符(+、-、*、/、\%、**等):
            c = 3 + 2       加法
            d = 3 ** 2      平方
            e = 3 \% 2      取余数
            f = "I Love" + "Python." 字符串连接,d的值为:I Love Python.
            python数值运算没有限制(仅有硬件内存限制位宽、CPU限制性能):
            g = 1234 ** 4321
        \item 语句:
            message = "Hello World"   赋值
            print(message)            输出
            源码:e1.py
    \end{itemize} 
\end{frame}

% 以下帧是模板
\begin{frame}
    \frametitle{第一张标题}
    第一张幻灯片
\end{frame}

\begin{frame}
    \frametitle{第二张标题}
    第二张幻灯片
\end{frame}

\begin{frame}
    \frametitle{演示暂停}
    \frametitle{第二张标题}
    我們先說明...  
    \pause 
    然後可以發現...  
    \pause 
    就是這樣分段! 
\end{frame}

\begin{frame}
    \frametitle{演示列表式暂停}
    \begin{itemize}
        \item 
            第一項 
            \pause
        \item 
            第二項 
            \pause 
        \item 
            第三項 
    \end{itemize} 
\end{frame}

\begin{frame}
    \frametitle{演示列表式暂停(简单)}
    \begin{itemize}[<+->]
        \item
            第一項
        \item
            第二項
        \item
            第三項
    \end{itemize} 
    %第二张中强调
    \alert<2>{第二張}才重要
\end{frame}

\begin{frame}
    \frametitle{演示重点}
    \begin{block}{小重點} 
        重點就是重點。 
    \end{block}

    \begin{alertblock}{大重點}
        特別重要的東西。 
    \end{alertblock} 
\end{frame}

\begin{frame}[fragile]
    \frametitle{演示代码}
    代码如下:
    \begin{verbatim} 
    for i in range(10): 
        print(i)
    \end{verbatim}
    带颜色的代码如下:
    \begin{semiverbatim} 
    for \alert{i} in range(10): 
        print(\alert{i})
    \end{semiverbatim} 
\end{frame}


\begin{frame}
    \frametitle{演示多栏}
    \begin{columns} 
        \begin{column}{5cm} 
            % 一個5cm的欄
            這是欄一。 
        \end{column}

        \begin{column}{5cm} 
            % 另一個5cm的欄 
            這是欄二。 
        \end{column}
    \end{columns}
\end{frame}

\end{document}

