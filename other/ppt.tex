\documentclass[xcolor=dvipsnames]{beamer}
%\usecolortheme[named=Brown]{structure}
\useoutertheme{infolines}
\usetheme[height=7mm]{Rochester} 
\setbeamertemplate{navigation symbols}{} 
\setbeamertemplate{items}[ball]
\usepackage{xeCJK}
\usepackage{makeidx}
\usepackage{multirow}
\usepackage{color}
\usepackage{graphicx}
\setCJKmainfont{AR PL UKai CN}

% 将日期变为中文格式
\renewcommand{\today}{\number\year 年 \number\month 月 \number\day 日}

\begin{document}

\title{xelatex幻灯片模板}
\author{彭鹏}
\institute{中原电子}

\date{\today}
\begin{frame}
    \titlepage
\end{frame}

\begin{frame}
    \frametitle{第一张标题}
    第一张幻灯片
\end{frame}

\begin{frame}
    \frametitle{第二张标题}
    第二张幻灯片
\end{frame}

\begin{frame}
    \frametitle{演示暂停}
    \frametitle{第二张标题}
    我們先說明...  
    \pause 
    然後可以發現...  
    \pause 
    就是這樣分段! 
\end{frame}

\begin{frame}
    \frametitle{演示列表式暂停}
    \begin{itemize}
        \item 
            第一項 
            \pause
        \item 
            第二項 
            \pause 
        \item 
            第三項 
    \end{itemize} 
\end{frame}


\begin{frame}
    \frametitle{演示列表式暂停(简单)}
    \begin{itemize}[<+->]
        \item
            第一項
        \item
            第二項
        \item
            第三項
    \end{itemize} 
    %第二张中强调
    \alert<2>{第二張}才重要
\end{frame}

\begin{frame}
    \frametitle{演示重点}
    \begin{block}{小重點} 
        重點就是重點。 
    \end{block}

    \begin{alertblock}{大重點}
        特別重要的東西。 
    \end{alertblock} 
\end{frame}

\begin{frame}[fragile]
    \frametitle{演示代码}
    代码如下:
    \begin{verbatim} 
    for i in range(10): 
        print(i)
    \end{verbatim}
    带颜色的代码如下:
    \begin{semiverbatim} 
    for \alert{i} in range(10): 
        print(\alert{i})
    \end{semiverbatim} 
\end{frame}


\begin{frame}
    \frametitle{演示多栏}
    \begin{columns} 
        \begin{column}{5cm} 
            % 一個5cm的欄
            這是欄一。 
        \end{column}

        \begin{column}{5cm} 
            % 另一個5cm的欄 
            這是欄二。 
        \end{column}
    \end{columns}
\end{frame}

\end{document}

