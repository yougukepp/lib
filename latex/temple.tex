\documentclass[12pt,a4paper]{article}

%\documentclass[12pt,a4paper]{report}
%\documentclass{book}
%\documentclass{article}

% tikz
\usepackage{tikz}

\usepackage{xeCJK}
\usepackage{makeidx}
\usepackage[colorlinks,linkcolor=red]{hyperref}
\usepackage{xcolor}
\usepackage{listings}
\lstset{numbers=left,
    numberstyle=\tiny,
    keywordstyle=\color{blue!70}, commentstyle=\color{red!50!green!50!blue!50},
    frame=shadowbox,
    rulesepcolor=\color{red!20!green!20!blue!20}
}
\usepackage{multirow}

%中文断行
\XeTeXlinebreaklocale "zh"
%段首缩进
\parindent 2em
\usepackage{indentfirst}
%文泉驿 字体
\setCJKmainfont{文泉驿等宽微米黑:style=Regular}
%\setCJKmainfont{文泉驿等宽正黑}
%楷体
%\setCJKmainfont{AR PL UKai CN}

% 将日期变为中文格式
\renewcommand{\today}{\number\year 年 \number\month 月 \number\day 日}

%制作索引
\makeindex
\printindex
\setcounter{secnumdepth}{5}

\title{标题}
\author{彭鹏}

\begin{document}
\maketitle
\tableofcontents
\newpage

\section{概述}
概述第一段

概述第二段
%\newpage
\section{枚举} 
枚举例子:

\begin{enumerate}
    \item 元素1
        内容1
    \item 元素2
        内容2
\end{enumerate}

\section{超链接} 
超链接:\href{http://ftp.kernel.org/pub/linux/kernel/v3.x/}{内核url}

\section{脚注} 
我是日期脚注\footnote{\today}

我是脚注\footnote{脚注内容}

\section{引用} 
我是一个位置\label{标签1}

制作空白页用于观察跳转
\newpage

我要引用第\pageref{标签1}页。

\section{pgf图片} 
下面是图片
\begin{tikzpicture}
    \draw (-1.5,0) -- (1.5,0);
    \draw (0,-1.5) -- (0,1.5);
\end{tikzpicture}

\section{表格} 
表格例子\ref{表格例子}
\begin{table}[!hbp]
\begin{center}
    \begin{tabular}{|l|l|}
        \hline
        表头列1 & 表头列2 \\
        \hline
        1行1列 & 1行2列 \\
        \hline
        2行1列 & 2行2列 \\
        \hline
        3行1列 & 3行2列 \\
        \hline
    \end{tabular}
    \caption{表格例子\label{表格例子}}
\end{center}
\end{table}

\section{原样引用}
使用目录树作为例子
\setlength{\unitlength}{1mm}
\begin{figure}[!hbp]
\begin{verbatim}
                         ppcc
                         |-- tar
                         |-- build
                         |   |-- src
                         |   `-- work
                         `-- tools
\end{verbatim}
\caption{目录树\label{目录树}}
\end{figure}

\end{document}

